\chapter*{Introduction}
\addcontentsline{toc}{chapter}{Introduction}
Research in high energy physics focuses on studying particles at the most fundamental level in order to discover how they interact. In order to gain an insight on the physical processes that involve matter at the subatomic scale, it is necessary to accelerate particles and make them collide at very high energies. The Large Hadron Collider (LHC)~\cite{LHC}, at CERN (Counseil Européen pour la Recherche Nucléaire)~\cite{CERN} accelerates protons to 14 TeV, collimating them in two parallel beams that collide at specific sites where the particle detectors are located. Four main experiments are hosted at the LHC facility: ALICE~\cite{ALICE}, LHCb~\cite{LHCb}, CMS~\cite{CMS} and ATLAS~\cite{ATLAS}, of which the latter two are general purpose experiments looking to improve our knowledge of the Standard Model~\cite{standard_model}. When protons collide at such high energies, many other particles are produced and scattered in every direction. These particles fly through the detectors and are stopped at different levels depending on their charge and energy, and can thus be identified. In high energy physics there are mainly two ways to look for new physics: either increase the energy at which particles collide, or observe more collisions to gain insight into the rarest processes. For this reason, LHC periodically undergoes a series of hardware upgrades that mainly  aim to address both the aforementioned points. In particular, the LHC is scheduled to receive a major upgrade by 2029, which will deal with both points. Firstly, the energy of the colliding beams will be increased up to the theoretical limit of the machine of 7 TeV per beam, with resulting collisions at 14 TeV in the center of mass. Secondly, the number of collisions observed per unit time (luminosity) will increase by roughly a factor 10. Because of this, detectors have to receive periodic hardware upgrades as well, to improve their sensitivity. As a result of the hardware upgrades, the statistics of each event increase as well as the amount of data to be collected and processed with very strict time requirements. To accomplish such a task, each experiment's software must be regularly updated together with the hardware. In particular, we can distinguish two kinds of algorithms used to process the data coming from the detectors: trigger and reconstruction algorithms. The first kind filters the collisions' data in real time by selecting and saving only events that satisfy particular conditions due to which they might be good candidates for new physics. Reconstruction algorithms are instead used to study these potentially interesting events by reproducing the chain of decays and interactions with the detectors that have occurred after the proton-proton collision. \newline
In this context, the CMS experiment has invested part of its resources to explore different heterogeneous computing solutions. This means that several parts of the software reconstruction can run simultaneously on different types of devices, like CPUs, GPUs or FPGAs. Recently, considerable efforts have been put in order to improve the code portability across different architectures and backends. Since writing multiple versions of the same code for each device would be extremely inefficient, some abstraction layers have been considered. In every case, the main objective is to produce an executable file that can run on many different architectures while maintaining a level of performance as close as possible to the native implementation. \newline
This thesis focuses on one of these abstraction layers, SYCL~\cite{sycl_specification}, which is based on the ISO C++ standard~\cite{iso_c++}. In particular, the porting experience of one CMS clustering algorithm from CUDA~\cite{CUDA} code, designed to run exclusively on NVIDIA GPUs, to SYCL, with a focus on the performance and physics analysis, is discussed. \newline
In Chapter~\ref{ch:1} the clustering problem in high energy physics is introduced, with a description of the CLUE algorithm, chosen for the porting experience. 
In Chapter~\ref{ch:2} the SYCL standard is presented, along with its various implementations. In particular, more details are given on Data Parallel C++, the chosen implementation to carry out the port.
Finally, in Chapter~\ref{ch:3} the porting results are shown, as well as the performance measurements.